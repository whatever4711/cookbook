% Complete recipe example
\begin{recipe}
[% 
    preparationtime = {\unit[90]{min}},
    %bakingtime={\unit[1]{h}},
    %bakingtemperature={\protect\bakingtemperature{
     %   fanoven=\unit[230]{\textcelcius},
     %   topbottomheat=\unit[195]{°C},
     %   topheat=\unit[195]{°C},
     %   gasstove=Level 2}},
    portion = {\portion{4}},
    %calory={\unit[3]{kJ}},
    source = {Köstlich Vegetarisch}
]
{Indischer-Linsen-Fenchel-Eintopf}
    
    \graph
    {% pictures
        %small=pic/CurlyKaleSoup2,     % small picture
        big=pic/LinsenFenchelEintopf  % big picture
    }
    
    %\introduction{%
     %   \blindtext
    %}
    
    \ingredients{%
        150g	& 	bunte oder helle Quinoa\\
		600g	& 	Brokkoli\\
		500g   	&	Süßkartoffeln\\
		2 		&	Knoblauchzehen\\
		1		&	Zwiebel\\
		10		&	Salbeiblätter\\
		4EL		&	Olivenöl\\
		1TL		&	Zucker\\
 		\unitfrac{1}{2}Bund & Petersilie\\
 		100g	&	Parmesan\\
 		3		&	Eier
 				&	Muskat\\
 	\unitfrac{1}{2}TL & Chili\\ 	
 				&	Salz und Pfeffer\\
    }
    
    \preparation{%
    \step Fenchel waschen, putzen, Fenchelgrün zur Seite legen, Knollen vierteln, Strunk entfernen und Viertel würfeln. Möhren ebenfalls würfeln. Chili waschen, der Länge nach halbieren, entkernen und fein würfeln. Ingwer schälen und klein hacken.
    \step Öl und Senfkörner in einem großen Topf leicht erhitzen, dabei Deckel drauflegen, die Senfkörner können springen. Ingwer und Chili zugeben und 1 Minute braten. Gemüse, Kurcuma und Curry zugeben, weitere 3 Minuten braten. Linsen und rund 600 ml Wasser zugeben. Zum Kochen bringen und im geschlossenem Topf auf kleiner Hitze ca 25 Minuten garen, bis Linsen weich aber noch bissfest sind. Mit Salz und Zimt würzen und 3-4 minuten durchziehen lassen.
    \step Inzwischen Reis nach Packunganleitung (in der Regel mit doppelter Menge Wasser) gar kochen. Fenchelgrün und Petersilie abrausen, trockenschütteln und fein hacken. Linsen-Fenchel-Eintopf mit Reis und Kräutern gariert servieren.
    }
    
    %\suggestion[Headline]
    %{%
    %    \blindtext
   % }
    
   % \suggestion{%
    %    \blindtext
   % }
    
    \hint{%
       Auch lecker: Am Ende noch ein paar Organgenscheiben unterheben und Curry mit Orangensaft abschmecken.
    }
    
\end{recipe}
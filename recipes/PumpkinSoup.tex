% Complete recipe example
\begin{recipe}
[% 
    preparationtime = {\unit[40]{min}},
    %bakingtime={\unit[1]{h}},
    %bakingtemperature={\protect\bakingtemperature{
     %   fanoven=\unit[230]{\textcelcius},
     %   topbottomheat=\unit[195]{°C},
     %   topheat=\unit[195]{°C},
     %   gasstove=Level 2}},
    portion = {\portion{4}},
    %calory={\unit[3]{kJ}},
    source = {Köstlich Vegetarisch}
]
{Fruchtige Butternut-Kürbis-Suppe}
    
    \graph
    {% pictures
        %small=pic/PumpkinSoup,     % small picture
        big=pic/PumpkinSoup  % big picture
    }
    
    %\introduction{%
     %   \blindtext
    %}
    
    \ingredients{%
        1,2 kg	& 	Butternut-Kürbis \\
		250 g	& 	Äpfel (säuerlich) \\
		100    	&	Zwiebeln \\
		3 EL	&	Butter \\
		2 TL	&	Curry \\
		40 g	&	Weizenmehl \\
		1 Liter	&	Gemüsebrühe \\
		1 		&	Bio-Orange \\
 				&	Salz und Pfeffer\\
 				&	Muskat \\
 		4 EL	& 	Cr{\'e}me fra{\^i}che
    }
    
    \preparation{%
    \newline
       \step Kürbis vierteln, schälen, fasriges Gewebe und Kerne entfernen. Äpfel vierteln, schälen und Kerngehäuse entfernen und Zwiebel schälen. Alles 1\unitfrac{1}{2} cm groß würfeln.
       \step Butter in großen Topf zerlassen und Zwiebeln darin anschwitzen. Kürbis, Äpfel und Currypulver hinzugeben und unter Rühren 3 Minuten anbraten. Mit Mehl bestäuben und mit Gemüsebrühe angießen. 15-20 Minuten mit geschlossenem Topf auf mittlerer Hitze kochen.
       \step Inzwischen Orangen heiß abwaschen, trocken reiben und Schale fein abreiben; Saft auspressen und beides zur Suppe geben. Aufkochen lassen, fein pürieren und mit Gewürzen abschmecken.   
    }
    
    \suggestion[Servierbeispiel]
    {%
        Suppe auf Wunsch mit Cr{\'e}me fra{\^i}che garnieren und servieren.
    }
    
   % \suggestion{%
    %    \blindtext
   % }
    
    \hint{%
        Die Suppe erhält durch Zugabe von Schwarzkümmel einen pfeffrigen Geschmack ohne zusätzliche Schärfe.
    }
    
\end{recipe}
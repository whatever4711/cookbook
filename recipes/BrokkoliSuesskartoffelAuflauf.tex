% Complete recipe example
\begin{recipe}
[% 
    preparationtime = {\unit[90]{min}},
    %bakingtime={\unit[1]{h}},
    %bakingtemperature={\protect\bakingtemperature{
     %   fanoven=\unit[230]{\textcelcius},
     %   topbottomheat=\unit[195]{°C},
     %   topheat=\unit[195]{°C},
     %   gasstove=Level 2}},
    portion = {\portion{4}},
    %calory={\unit[3]{kJ}},
    source = {Köstlich Vegetarisch}
]
{Brokkoli-Süßkartoffel-Auflauf mit Quinoa-Haube}
    
    \graph
    {% pictures
        small=pic/BrokkoliMitQuinoaHaube,     % small picture
        big=pic/BrokkoliMitQuinoaHaube2  % big picture
    }
    
    %\introduction{%
     %   \blindtext
    %}
    
    \ingredients{%
        150g	& 	bunte oder helle Quinoa\\
		600g	& 	Brokkoli\\
		500g   	&	Süßkartoffeln\\
		2 		&	Knoblauchzehen\\
		1		&	Zwiebel\\
		10		&	Salbeiblätter\\
		4EL		&	Olivenöl\\
		1TL		&	Zucker\\
 		\unitfrac{1}{2}Bund & Petersilie\\
 		100g	&	Parmesan\\
 		3		&	Eier\\
 				&	Muskat\\
 	\unitfrac{1}{2}TL & Chili\\ 	
 				&	Salz und Pfeffer\\
    }
    
    \preparation{%
    \step Quinoa in Sieb kalt abrausen. In reichlich Salzwasser ca 18 Minuten bei kleiner Hitze im halb offenen Topf garen. Abgießen abtropfen und lassen.
    \step Inzwischen Brokkoli putzen, in kleiner Röschen zerteilen, dicke Stiele schölen und in kleine Stücke schneiden, alles waschen. Salzwasser in Topf zum Kochen bringen und Stiele darin 3 Minuten blanchieren. Röschen zugeben und weitere 3 Minuten blanchieren danach abgießen. Abgießen und gut abtropfen lassen. Süßkartoffeln schälen und würfeln.
    \step Knoblauch und Zwiebeln schälen und würfeln. Salbeiblätter kalt abbrausen, tocken schütteln und in feine Streifen schneiden. Öl in Pfanner erhitzen und Knoblauch, Zwiebeln und Salbei unter rühren darib ca 3 Minuten anbraten. Süßkartoffeln und zucker zugeben und weitere 3 Minuten anbraten. Mit 150ml Wasser ablöschen und weitere 10 Minuten garen (bis Wasser verdampft ist). Etwas abkühlen lassen und mit Brokkoli vermischen in eine Auflaufform. Petersilie abrausen, trockenschütteln und fein hacken und ebenfalls unterheben. Mit Salz und Pfeffer würzen.
    \step Backeofen auf 180 \textcelcius C (Umluft 160 \textcelcius C) vorheizen. Käse reiben. Eier mit Schneebesen verquirlen und mit Käse zu Quinoa geben. Mit Salz, Muskat und Chili würzen gt vermischen. Quinoa masse auf gemüse verteilen und  ca. 35-40 Minuten backen und warm servieren.
    }
    
    %\suggestion[Headline]
    %{%
    %    \blindtext
   % }
    
   % \suggestion{%
    %    \blindtext
   % }
    
  
    
\end{recipe}
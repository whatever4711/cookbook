
\begin{recipe}
[% 
    preparationtime = {\unit[45]{min}},
    bakingtime={\unit[20--30]{min}},
    bakingtemperature={\protect\bakingtemperature{
        fanoven=\unit[180]{\textcelcius}}},
    source = {Lieblingsrezept von Daniel (\`a la Mama)}
]
{Lasagne Bolognese}
    
    
	\ingredients{%
				&	\textbf{Bolognese-Soße}\\
		500g 	& 	Hackfleisch\\
		1		& 	Zwiebel\\
				&	Tomatenmark\\
				&   Gewürze (Petersilie, Basilikum, Thymian, Muskat)\\
				&	\\
				&	\textbf{Béchamel-Soße}\\
		80g		&	Mehl\\
		80g		&	Butter\\
		500ml	&	Milch\\
		500ml	& 	Sahne\\
	}
    
    \preparation{%
    \newline
	    \step \textbf{Bolognese-Soße}\\	    
	    Zwiebeln schälen, hacken und in der Pfanne in etwas Olivenöl andünsten. Dann Hackfleisch dazugeben und anbraten, salzen und pfeffern.
	    Nun mit ca. 1 Tasse Wasser aufgiessen, Tomatenmark und Gewürze hinzugeben, abschmecken und ca. 30 Minuten ziehen lassen.
	    \step \textbf{Béchamel-Soße}\\	    
	    Butter schmelzen, Mehl einrühren (peu a peu, brent leicht an, darf aber schon etwas rösten), Milch dazu geben und Masse aufkochen, dabei ständig umrühren (brennt immernoch leicht an!) mit Salz, weißem Pfeffer und Muskat würzen.
	    \step \textbf{Lasagne zusammenschichten}\\	    
	    Ofen vorheizen. Form leicht buttern. Schicht Béchamel-Sauce, Schicht Teigplatten, Schicht Bolognese-Sauce (dünn), und dann wieder Teigplatten usw. Wenn kein Platz mehr in der Form, die letzte Schicht Béchamel-Sauce mit Parmesan bestreuen und ca. 20--30 Minuten in den Ofen.
    }
    
	\hint{%
		Dazu passt italienischer oder auch schwerer Rotwein.   
    }
    
\end{recipe}
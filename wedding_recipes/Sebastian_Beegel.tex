% Complete recipe example
\begin{recipe}
[% 
    preparationtime = {\unit[45]{min}},
    portion = {\portion{4}},
    source = {Lieblingsrezept von Janina und Sebastian}
]
{Beegel mit Honigfrikadellen und Omelett}
    
    \graph
    {% pictures
        %small=pic/PumpkinSoup,     % small picture
       big=pic/Sebastian_Honigfrikadellen.jpg % big picture
    }
    
    %\introduction{%
     %   \blindtext
    %}
    
    \ingredients{%
        50 g & 	Joghurtsalatcreme\\
	50 g	& 	mittelscharfer Senf \\
		2 EL&	Ahornsirup \\
		80 g &	Cheddarkäse\\
		2 &Tomaten\\
		4 Stiele&Petersilie\\
		2 Scheiben&Toastbrot\\
		1 & Zwiebel\\
		1 Zweig & Rosmarin (klein)\\
		350 g & gemischtes Hack\\
		4 & Eier (Gr. M)\\
		2 EL & Honig\\
		2 EL & Milch\\
		1 EL & Butter\\
		1--2 EL & Öl\\
		4 & Bagels (à ca. 150 g)\\
	& Salz\\
	& Pfeffer\\
 }
    
    \preparation{%
    \newline
       \step Für den Dip Salatcreme, Senf und Ahornsirup verrühren. Mit Salz abschmecken. Käse grob raspeln. Tomaten waschen und in Scheiben schneiden. Petersilie waschen, trocken schütteln und hacken. 
       \step Für die Frikadellen Toast in kaltem Wasser einweichen. Zwiebel schälen und sehr fein würfeln. Rosmarin waschen, trocken tupfen, Nadeln abzupfen und fein hacken. Brot ausdrücken. Mit Hack, Zwiebel, Rosmarin, 1 Ei, Honig, 1 TL Salz und \unitfrac{1}{2} TL Pfeffer in einer Schüssel gut verkneten. 
\step Daraus 4 große, flache Frikadellen formen. 
\step 3 Eier und Milch verquirlen. Mit Salz und Pfeffer würzen. Butter in einer beschichteten Pfanne erhitzen. Eimasse darin bei schwacher Hitze stocken lassen, dabei mit einem Pfannenwender etwas zusammenschieben.  
\step  Herausnehmen, vierteln. Öl im Bratfett erhitzen. Frikadellen darin von jeder Seite ca. 4 Minuten braten.
\step Bagels waagerecht aufschneiden. Unterhälften mit etwas Dip bestreichen, jeweils mit 1 Frikadelle, 1 Stück Omelett und Tomaten belegen und mit Petersilie und Käse bestreuen. Darauf wieder etwas Dip geben, Deckel daraufsetzen.
    }
    
   \suggestion[Servierbeispiel]
   {%
Dazu passen Pommes Frites oder Süßkartoffel-Fries:\\
Kartoffeln schälen, gründlich waschen und in grobe Stifte schneiden. Kartoffeln, 3--4 EL Ahornsirup, 2 EL Öl und 1 TL Salz mischen. Auf ein mit Backpapier ausgelegtes Backblech verteilen. Im vorgeheizten Backofen (E-Herd: 200 °C / Umluft: 175 °C) 30--40 Minuten backen. Kartoffelsticks aus dem Ofen nehmen, eventuell mit groben Salz bestreuen, dazureichen.
    }
    
   % \suggestion{%
    %    \blindtext
   % }
    
%  \hint{%
 %              Auch lecker mit Kokosmilch anstelle von Sahne und Curry anstelle von
%Kreuzkümmel.\\
  %             Rote Bete im Schnellkochtopf etwa 7 Minuten (je nach Größe) garen und erst
%dann schälen und schneiden.
 %   }
    
\end{recipe}
\begin{recipe}
[% 
    preparationtime = {\unit[45]{min}},
    portion = {\portion{2}},
    source = {Ein weiterer Klassiker aus Martins \& Linus' Kochstudio, adaptiert von Chefkoch},
]
{Linus' \& Martins Risotto}
        
    \ingredients{%
		300g	&	Risottoreis\\
		1		& 	Schalotte, klein geschnitten\\
		1 \unitfrac{1}{2} l &	Gem"usebrühe\\
		40g		&	Butter\\
		150ml	&	Weißwein\\
		1 Dose	&	Safranfäden\\
		50g		&	Butter, kalt\\
		150g	&	Parmesan, frisch gerieben\\
				&	Salz\\
				&	Pfeffer
    }
    
    \preparation{%
    \newline
    \step Die Brühe wird in einem Topf heiß gehalten, ohne zu kochen und dann folgt der 1. Schritt, das Anschwitzen -- \textit{solfriggere}. Die Schalotte wird ganz sanft, ohne zu bräunen in 5 Minuten in der Butter angeschwitzt. 
    
    \step Für die 2. Stufe wird der Reis zugegeben und so oft gewendet, bis jedes Korn von der Butter benetzt ist. Dieser Vorgang nennt sich \textit{tostare}. Jetzt wird die Temperatur nach Fingerspitzengefühl leicht erhöht und der Wein angegossen, der ganz verdampfen muss. Nun fügt man den Safran zu. 
    
    \step Schon ist man in der 3. Stufe, dem eigentlichen Kochen, dem \textit{cuocere}, in dem kellenweise die heiße Brühe zugegeben wird. Dieser Vorgang dauert 17 -- 18 Minuten, um einen bissfesten und cremigen Risotto zu kochen. Dabei sollte permanent umgerührt und die Körner vom Rand und Boden des Topfes geschabt werden. 
    
	Die Temperatur muss die Brühe gerade eben kochen lassen und konstant bleiben. Ist die Brühe fast eingekocht, wird die nächste Kelle angegossen. Ab der 14. Minute muss man aufpassen, nicht mehr zu viel Brühe anzugießen, damit der Reis zum Schluss nicht zu flüssig ist. Am Ende des Kochens wird die Hitze deutlich reduziert, um den Reis für 1 Minute ruhen zu lassen. 
    
    \step Dann folgt der 4. und letzte Schritt, die \textit{Mantecatura}, das Verrühren mit kalten Butterwürfeln und frisch geriebenem Parmesan. Nicht vergessen, mit Salz und Pfeffer abzuschmecken. Jetzt sollte die perfekte Konsistenz erreicht sein, die man \textit{Risotto all´onda} nennt. Das bedeutet, wenn man den Topf zur Seite kippt, schlägt der Reis Wellen. 
       
    }
    
    \suggestion[Servierbeispiel]
    {%
		Den Risotto in tiefen Tellern servieren und möglichst bald essen, sonst verliert er seine Konsistenz. Traditionell wird Risotto aber immer ohne Beilagen und erst recht nicht selbst als Beilage gereicht.
    }
    

    \hint{%
       Mehr K"ase ist mehr!
    }
    
\end{recipe}
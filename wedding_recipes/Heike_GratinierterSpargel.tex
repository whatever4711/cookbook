% Complete recipe example
\begin{recipe}
[% 
    preparationtime = {\unit[65]{min}},
    bakingtime={\unit[20]{min}},
    bakingtemperature={\protect\bakingtemperature{
        topbottomheat=\unit[200]{°C},
       gasstove=Stufe 3}},
    portion = {\portion{4}},
    source = {Lieblingsrezept von Heike und Sarah}
]
{Gratinierter Spargel}
    
    
    \ingredients{%
    &\textbf{Für den Teig:} \\
        100 g	& 	Mehl \\
		\unitfrac{1}{4} Liter	& 	Milch \\
		2 &	Eier (Größe M) \\
			& \textbf{Für die Füllung:}\\
		1 kg	&	Spargel \\
		6 & Tomaten\\
		 \unitfrac{1}{2} Bund & Oregano\\
		 1 & Zwiebel\\
		5 EL & Olivenöl\\
		2-3 EL& Balsamessig\\
		1 Prise & Zucker\\
		& \textbf{Für die Sauce:}\\
		2 EL & Butter\\
		1 geh. EL & Mehl\\
		250 ml & Gemüsebrühe\\
		250 ml & Schlagsahne\\
		& \textbf{Außerdem:}\\
					&	Salz\\
			&	Muskatnuss \\
					&Pfeffer\\
		100 g & geriebener Parmesankäse\\
		60 g & Pinienkerne\\
    }
    
    \preparation{%
    \newline
       \step Mehl mit Milch und Eiern zu einem glatten Teig verquirlen, mit Salz und Muskat würzen. Zugedeckt beiseite stellen.
       \step Spargel waschen, die Enden abschneiden. Spargel in kochendem Salzwasser 6-8 Minuten garen, abtropfen lassen. Tomaten überbrühen, häuten, halbieren und entkernen. Fruchtfleisch fein würfeln. Oregano waschen, trockenschütteln und fein hacken. Zwiebel abziehen, fein würfeln. Im heißen Öl glasig dünsten. Vom Herd nehmen. Tomaten unterrühren. Mit Oregano, Essig, Salz, Pfeffer, Zucker abschmecken.
       \step Aus dem Teig im heißen Fett 4 Pfannkuchen backen, abkühlen lassen. Für die Sauce Butter erhitzen. Mehl einrühren. 1 Minute anschwitzen. Brühe angießen, einrühren, nochmal aufkochen. Sahne einrühren, nochmal aufkochen. Mit Salz, Pfeffer und Muskat kräftig abschmecken.
       \step Elektro-Ofen auf 200 Grad vorheizen. In jeden Pfannkuchen \unitfrac{1}{4} des Spargels und der Tomaten einrollen. In eine flache, gefettete Form legen. Mit Sauce begießen. 50 g geriebenen Käse und die Pinienkerne überstreuen. Im Ofen bei 200 Grad (Gas Stufe 3) 15-20 Minuten backen. Mit Parmesan bestreut servieren.
    }
   \hint{%
               Getränketipp: dazu passt leichter Rotwein, z.B. Valpolicella
    }
    
\end{recipe}
% Complete recipe example
\begin{recipe}
[% 
    preparationtime = {\unit[50]{min}},
    bakingtime={\unit[50]{min}},
    portion = {\portion{4}},
    source = {Ein weiteres Lieblingsrezept von Janina und Sebastian}
]
{Scharfe Erdnuss-Currysoße zu Schweinefilet}
    
    \graph
    {% pictures
        %small=pic/PumpkinSoup,     % small picture
       big=pic/Sebastian_Schweinefilet.jpg % big picture
    }
    
    %\introduction{%
     %   \blindtext
    %}
    
    \ingredients{%
        2 & 	Knoblauchzehen\\
	2	& 	Schalotten \\
		1 Stück&	Ingwer (ca. 20 g) \\
		1 &	Chilischote\\
		2 EL&geröstete Erdnüsse\\
		2&Schweinefilets (à ca. 350 g)\\
		3 EL&Öl\\
		2 EL & Tomatenmark\\
		3 EL & Erdnussbutter (Glas)\\
		2--3 TL & rote Currypaste \\
		400 ml & ungesüßte Kokosmilch\\
		3--4 EL & Sojasoße\\
		250 g & Zuckerschoten\\
		2--3 Stiele & Koriander\\
	& Salz\\
	& Pfeffer\\
	&brauner Zucker\\
 }
    
    \preparation{%
    \newline
       \step Für die Soße Knoblauch und 1 Schalotte schälen und fein würfeln. Ingwer schälen und fein reiben. Chili längs einschneiden, entkernen, waschen und hacken. Erdnüsse hacken.
       \step Ofen vorheizen (E-Herd: 150°C / Umluft: 125°C / Gas: s. Hersteller). Fleisch trocken tupfen. 2 EL Öl in einer großen Pfanne erhitzen und darin rundherum kräftig anbraten. Mit Salz und Pfeffer würzen. 
\step Auf ein Backblech legen und im heißen Ofen 10--12 Minuten braten. 
\step  Bratfett in der Pfanne erhitzen. Knoblauch, Schalottenwürfel, Ingwer und Chili darin andünsten, bis es duftet. Mit 1 TL braunem Zucker bestreuen und kurz karamellisieren. Tomatenmark, Erdnussbutter und Currypaste einrühren, kurz anschwitzen.
\step Kokosmilch, 100 ml Wasser und Sojasoße angießen. Aufkochen und 8--10 Minuten köcheln. Soße warm halten.
\step Die Zuckerschoten putzen, waschen und in Stücke schneiden. 1 Schalotte schälen und fein würfeln. 1 EL Öl erhitzen. Schalotte darin andünsten. Zuckerschoten und 2--3 EL Wasser zugeben und darin 2--3 Minuten garen. 
\step Koriander waschen, trocken schütteln und Blätter abzupfen. Fleisch herausnehmen und kurz ruhen lassen. Filet aufschneiden. Soße aufkochen. Alles anrichten und mit Koriander bestreuen.
    }
    
   \suggestion[Servierbeispiel]
   {%
Dazu schmeckt Reis.
    }
    
   % \suggestion{%
    %    \blindtext
   % }
    
%  \hint{%
 %              Auch lecker mit Kokosmilch anstelle von Sahne und Curry anstelle von
%Kreuzkümmel.\\
  %             Rote Bete im Schnellkochtopf etwa 7 Minuten (je nach Größe) garen und erst
%dann schälen und schneiden.
 %   }
    
\end{recipe}
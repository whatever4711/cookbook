\begin{recipe}
[% 
    preparationtime = {\unit[3 \unitfrac{1}{2}]{h}},
    bakingtime={\unit[2]{h}},
    portion = {\portion{6}},
    source = {Lieblingsrezept von Chris (original von chefkoch.de)}
]
{Eisbein gebraten}
    
    
    \ingredients{%
        1 \unitfrac{1}{2} kg	& 	Eisbein \\
		1 Zehe	& 	Knoblauch, gehackt \\
		1  &	mittelgroße Zwiebel, geviertelt \\
		1	&	Lorbeerblatt\\
		5 Körner	&	Piment \\
		5 Körner	&	Pfeffer \\
		\unitfrac{1}{2} TL & Kümmel \\
		2 TL & Salz\\
		 1 Liter & Wasser\\
    }
    
    \preparation{%
    \newline
     Den Backofen auf etwa 200° C vorheizen. 
       \step Ein großes Eisbein in einen Bräter legen, die Gewürze und den gehackten Knoblauch zugeben. Die Zwiebel schälen, vierteln, auch mit rein. Mit Wasser auffüllen, Salz nicht vergessen. 
       \step Nun den Bräter für 2 Stunden in den Backofen. Zwischendurch immer mal etwas Wasser nachgießen. Gegen Bratende soll aber mindestens noch 1/4l Flüssigkeit übrig sein.
       \step Wenn das Fleisch weich ist, vom Knochen entfernen. Vorsichtig, dass die Haut nicht abfällt. Die Fleischstücke mit der Schwarte nach oben nochmal in den Bräter und ohne Deckel etwa 20 min bräunen. 
    }
    
    \suggestion[Servierbeispiel]
    {%
       Passt zu Sauerkraut und Kartoffeln, aber auch zu Erbsenmus. 
	}
    
   \hint{%
        Wer das Fleisch gerne rosa möchte, sollte es 12 h vorher mit ein wenig Pökelsalz einreiben. 
    }
    \hint{Für einen noch würzigeren Sud können einige Babymöhren, \unitfrac{1}{2} Sellerieknolle, 4 TL Instantbrühe und etwas mehr Pfeffer, Lorbeerblätter und Nelken mitgebraten werden.}
    
\end{recipe}

\begin{recipe}
[% 
    preparationtime = {\unit[30]{min}},
    bakingtime={\unit[80]{min}},
    bakingtemperature={\protect\bakingtemperature{
        fanoven=\unit[170]{\textcelcius}}},
    source = {Lieblingskuchen von Linus}
]
{Bierkuchen}
        
    
    \ingredients{%
        100g	& Butter \\	
		250g    & Zucker \\
		2		& Eier \\
		1 TL	& Zimt \\
		1 TL	& Nelken \\
		100g 	& Zitronat\\
		100g 	& Orangeat\\
		250g	& Sultaninen\\
		375g	& Mehl\\
		\unitfrac{1}{2} Seidla & Bier, nicht zu hopfig \\
		1 TL	& Natron\\
    }
    
    \preparation{%
    \newline
       \step Butten weich werden lassen, Natron im Bier aufl"osen und zusammen mit den anderen Zutaten gut verr"uhren. Nebenbei vom restlichen Bier trinken -- R"uhren ist nunmal anstrengend.
       \step Masse in eine gro\ss e, unter Umst"anden gefettete Kastenkuchenform einf"ullen.
       \step Backen. Dauert lange, aber man kann sich ja noch ein weiteres Bier aufmachen.
       \step Nach dem Backen ausk"uhlen lassen und mit Puderzucker bestreuen.
    }
    \suggestion{Es bietet sich an den Kuchen "uber Nacht stehen zu lassen, damit er sein Aroma optimal entfalten kann.}

    \hint{%
       Der Kuchen b"ackt zwar sehr lange, daf"ur ist er aber auch "uber eine Woche lang haltbar -- sofern er so "uberhaupt lange "uberlebt. 
    }
    
\end{recipe}
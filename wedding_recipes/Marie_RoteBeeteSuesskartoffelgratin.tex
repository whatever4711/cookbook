\begin{recipe}
[% 
    bakingtime={\unit[15-20]{min}},
    bakingtemperature={\protect\bakingtemperature{
        fanoven=\unit[160-170]{\textcelcius}}},
    source = {Lieblingrezept von Marie und Andreas}
]
{Rote Beete-Süßkartoffelgratin mit Ingwer}
    
    \graph
    {% pictures
       big=pic/Marie_RotebeteSuesskartoffelgratin % big picture
    }

    
    \ingredients{%
        750 g& 	Rote Bete\\
	1	& 	große Süßkartoffel \\
		175 g&	geriebener Käse \\
		1 Stück&	Ingwer\\
		200 g &Schmand\\
		200 ml&Schlagsahne\\
		1&Knoblauchzehe\\
	& Salz\\
	& Pfeffer\\
	 & Kreuzkümmel\\
	&Zucker \\
 }
    
	\preparation{%
    \newline
	    \step Rote Bete schälen und in sehr kleine Stücke schneiden, sie braucht sonst sehr lange bis sie gar ist. In einem Topf bei geschlossenem Deckel ca. 20--25 Minuten auf mittlerer Stufe garen.
	    \step Mit der Süßkartoffel genauso verfahren, jedoch in etwas größere Stücke schneiden. Garzeit beträgt ca. 10--15 Minuten.
		\step Während das Gemüse köchelt, die ``Sauce'' vorbereiten. Hierzu hacken Sie Knoblauch und ein etwa daumennagelgroßes Stück Ingwer ganz klein, mischen es mit dem Schmand und der Sahne und schmecken alles kräftig mit Salz, Pfeffer, Zucker und Kreuzkümmel ab.
		\step Schütten Sie das gegarte Gemüse ab. Backofen vorheizen.
		\step Rote Bete und Süßkartoffel abwechselnd in eine Auflaufform geben, die Sauce darüber gießen und den Käse darauf verteilen.
		\step Im vorgeheizten Backofen bei ca. 160--170°C Umluft ca. 15--20 Minuten goldgelb backen.
    }
    
    
  \hint{%
	  Auch lecker mit Kokosmilch anstelle von Sahne und Curry anstelle von Kreuzkümmel.\\
      Rote Bete im Schnellkochtopf etwa 7 Minuten (je nach Größe) garen und erst dann schälen und schneiden.
    }
\end{recipe}
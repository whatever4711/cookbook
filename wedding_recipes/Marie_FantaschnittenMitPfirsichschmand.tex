\begin{recipe}
[% 
    bakingtime={\unit[25]{min}},
    bakingtemperature={\protect\bakingtemperature{
        fanoven=\unit[160]{\textcelcius}}},
    source = {Lieblingskuchen von Marie und Andreas}
]
{Fantaschnitten mit Pfirsichschmand}
    
    \graph
    {% pictures
       small=pic/Marie_FantaschnittenMitPfirsichschmand % big picture
    }
    
    
    \ingredients{%
    &\textbf{Für den Teig:}\\
        4	& 	Eier (M) \\
	250 g	& 	Zucker \\
		1 Pck.&	Vanillinzucker \\
		125 ml&	Öl \\
		150 ml & Limonade (Fanta)\\
		250 g&Mehl\\
		3 TL&Backpulver\\
		&\textbf{Für den Belag:}\\
	2 Dosen& Pfirsich (Abtropgewicht 470 g)\\
	600 ml & Schlagsahne\\
	3 Pck. & Sahnesteif\\
	5 Pck. & Vanillinzucker\\
	500 g & Schmand\\
	& Zimt und Zucker zum Bestreuen\\    }
    
    \preparation{%
    \newline
       \step Für den Teig Eier, Zucker ud Vanillinzucker schaumig schlagen. Öl und Fanta unterrühren. Mehl und Backpulver mischen und unterrühren.
       \step Den Teig auf ein gefettetes Backblech streichen und in den Backofen geben. Im vorgeheizten Backofen bei 160 Grad (Heißluft) ca. 25 Minuten backen lassen.
       \step Den Kuchen auf dem Blech erkalten lassen.
       \step Für den Belag Pfirsiche abtropfen lassen und in kleine Stücke schneiden. Sahne mit Sahnesteif und 3 Päckchen Vanillinzucker steif schlagen. Schmand mit den restlichen 2 Päckchen Vanillinzucker verrühren. Pfirsichstücke unter den Schmand rühren und die Sahne unterheben.
       \step Die Masse gleichmäßig auf den Kuchen streichen und mit Zimtzucker bestreuen.
    }
    

   \hint{%
	   Anstelle von Pfirsichen können auch 2 Dosen Mandarinen verwendet werden.
   }
    
\end{recipe}
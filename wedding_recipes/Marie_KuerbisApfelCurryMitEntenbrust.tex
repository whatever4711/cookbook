\begin{recipe}
[% 
    bakingtime={\unit[10]{min}},
    bakingtemperature={\protect\bakingtemperature{
        fanoven=\unit[160]{\textcelcius},
        topbottomheat=\unit[180]{°C},
     %   topheat=\unit[195]{°C},
       gasstove=Level 3}},
    portion = {\portion{4}},
    source = {Lieblingsrezept von Marie und Andreas}
]
{Kürbis-Apfel-Curry mit Entenbrust}
    
    \graph
    {% pictures
       big=pic/Marie_KuerbisApfelCurryMitEntenbrust  % big picture
    }
    
    \ingredients{%
        1	& 	Hokkaido-Kürbis (etwa 900g) \\
	4	& 	Schalotten \\
		1 &	rote Paprika \\
		1&	Knoblauchzehe \\
		1 & rote Chilischote\\
		4 EL &Öl\\
		&Salz\\
	2 & Sternanis\\
	1 TL & Currypaste (gelb)\\
	400 ml & Kokosmilch (Dose)\\
	1 Bund & Lauchzwiebeln\\
	4 & Äpfel (säuerlich, z.B. Boskop)\\
	2 & Entenbrustfilets (à 250-300g mit Haut, am besten Bio)\\
	2 TL & 5-Gewürze Pulver\\
	\unitfrac{1}{2}&Limette\\
	&frischer Pfeffer
    }
    
    \preparation{%
    \newline
       \step Kürbis und Paprika putzen, abspülen und in Würfel schneiden. Ingwer, Schalotten und Knoblauch
schälen. Ingwer und Knoblauch fein würfeln, Schalotten in Ringe schneiden. Chili abspülen und
fein hacken (mit Küchenhandschuhen arbeiten).
       \step Öl erhitzen, Kürbis portionsweise darin anbraten, salzen und herausnehmen. Ingwer,
Schalotten, Knoblauch, Chili und Sternanis in den Topf geben und andünsten.
       \step Paprika dazugeben und 1 Minute anbraten. Die Currypaste zufügen und ebenfalls 1 Minute
anbraten. Kokosmilch und Kürbisstücke zufügen. Zugedeckt 5--10 Minuten kochen. Lauchzwiebeln
putzen, abspülen, schräg in Ringe schneiden. Äpfel abspülen, vierteln, entkernen und in Spalten
schneiden.
       \step Backofen auf 180 Grad, Umluft 160 Grad, Gas Stufe 3 vorheizen.
       \step Entenbrust abspülen, trocknen, die Haut schräg bis in das Fettgewebe einritzen. Jede Entenbrust
auf der Hautseite mit je 1 TL 5-Gewürz-Pulver und 1⁄2 TL Salz einreiben. Das restliche Öl in einer
beschichteten Pfanne erhitzen, die Filets zuerst auf der Hautseite darin scharf anbraten, dann
wenden und noch 2 Minuten auf der Fleischseite braten. In Alufolie wickeln und im heißen Ofen
etwa 5-10 Minuten nachgaren lassen.
	\step Limettensaft auspressen. Die Apfelspalten in die Pfanne geben und im Entenbratfett goldbraun
anbraten. Zusammen mit den Lauchzwiebeln zum Curry geben, mit Salz, Pfeffer und Limettensaft
abschmecken.
	\step Entenbrustfilets aus der Folie nehmen und den aufgefangenen Fleischsaft aus der Folie zum
Curry geben. Fleisch in Scheiben schneiden, zusammen mit dem Curry servieren.
    }
    
   \suggestion[Servierbeispiel]
   {%
		Dazu Basmati-Reis und Koriander-Pesto!
    }
\end{recipe}
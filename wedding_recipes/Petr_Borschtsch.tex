\begin{recipe}
[% 
    preparationtime = {\unit[70]{min}},
    portion = {\portion{12}},
    source = {Lieblingsrezept von Petr}
]
{Borschtsch}

    
    \introduction{
Der Borschtsch ist weltweit bekannt. Es ist immer noch nicht klar, ob der Borschtsch ukrainisch oder russisch ist. Eigentlich ist es ja egal. Hauptsache, dass es schmeckt! \\
Die Prozesse des Fleischkochens und Vorbereitung der Gemüse lassen sich sehr gut parallelisieren. Den Borschtsch kann man auch vegetarisch machen, in dem das Fleisch auf dem Zubereitungsprozess eliminiert wird. \\
Die Anzahl der Zutaten bezieht sich auf 4-Liter-Topf.
    }
    
    \ingredients{%
    	1,0 kg & Rindfleisch (ein paar Knochen wären ganz gut für den Geschmack) \\
		500 g  & Kartoffeln \\
		300 g  & Gemüsekohl \\
		400 g  & rote Bete \\
		200 g  & Möhren \\
		200 g  & Zwiebeln \\
		3 El   & Tomatenpaste \\
		1 Tl   & Essig 6\% \\
		2–3    & Knoblauchzehen \\ 
		2–3    & Lorbeeren \\
	           & Salz \\
			   & Pfeffer \\
	           & Sonnenblumenöl \\
	           &\textbf{Dazu optional:}\\
	           &Schmand \\
&		Petersilie\\
&		Roggenbrot \\
&		Meerrettichaufstrich\\
&		Vodka\\
    }


    
    \preparation{%
    \newline
		\step Zunächst soll Rindfleisch 1.5 Stunden im Topf aufgekocht werden.
		\step Danach das Fleisch rausnehmen, würfeln und wieder in den Topf fügen.
		\step Zwiebel würfeln.
		\step Möhren auf der mittleren Oberfläche reiben.
		\step Gemüsekohl und rote Bete in Julienne schneiden.
		\step Rote Bete mit Sonnenblumenöl anbraten.
		\step Essig und Tomatenpaste hinzufügen und 5-7 Minuten dämpfen lassen.
		\step Zwiebel anbraten, nach ein paar Minuten Möhren hinzufügen. 
		\step Kartoffel würfeln. 
		\step In die kochende Brühe Kartoffel hinzufügen und anschließend salzen.
		\step Wenn die Brühe anfängt, wieder zu kochen, Gemüsekohl hinzufügen und 5 Minuten auf den mittleren Hitze lassen.
		\step Rote Bete hinzufügen und 10 Minuten aufkochen. 
		\step Möhren und Zwiebel hinzufügen. 
		\step Danach Lorbeeren hinzufügen und wenn nötig salzen und pfeffern. 
		\step Knoblauch pressen und dazu fügen. 
		\step Für ca. 20 Minuten ruhen lassen.     
    }
    
    \suggestion[Servieren]
    {%
        Der Borschtsch wird auf dem Teller mit Schmand und Petersilie serviert.
    }
    
    \hint{%
        Dazu passt ganz gut ein guter Shot Vodka und ein Stück Roggenbrot mit Meerrettich Aufstrich oben drauf!
    }
    
\end{recipe}
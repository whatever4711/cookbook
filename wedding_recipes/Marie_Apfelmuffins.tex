\begin{recipe}
[% 
    preparationtime = {\unit[15]{min}},
    bakingtime={\unit[17]{min}},
    bakingtemperature={\protect\bakingtemperature{
        fanoven=\unit[175]{\textcelcius}}},
    source = {Lieblingsmuffins von Marie und Andreas (original chefkoch.de)}
]
{Apfel-Muffins mit Rosinen und Nüssen}
    
    
    \ingredients{%
        2 \unitfrac{1}{2}	Tassen& 	Speisekleie (Haferkleie, keine Haferflocken)\\
	\unitfrac{1}{4} Tasse	& 	brauner Zucker \\
		\unitfrac{1}{4} TL&	Zimt \\
		1 EL&	Backpulver\\
		\unitfrac{1}{4} Tasse &Walnüsse oder Pecanüsse, gehackt\\
		\unitfrac{1}{4} Tasse&Rosinen\\
		1&mittelgroßer Apfel, fein geraspelt\\
	\unitfrac{3}{4} Tasse& Milch\\
	\unitfrac{3}{4} Tasse& Dicksaft (Apfeldicksaft)\\
	2 & kleiner Eier\\
	2 EL&Öl 
 }
    
    \preparation{%
    \newline
       \step Die trockenen Zutaten mit dem Apfel in einer Schüssel vermischen. Milch, Apfeldicksaft, Eier und Öl in einer anderen Schüssel oder im Mixer mischen. Zu den trockenen Zutaten geben und kurz vermengen.
       \step Ein Muffinblech einfetten oder mit Papierförmchen auslegen und den Teig einfüllen. 17 Minuten bei 175 Grad Umluft backen. Abkühlen lassen und genießen.
    }
    
  \hint{%
               Wenn man keinen Apfeldicksaft hat, kann auch eine halbe Tasse Apfelsaft verwendet werden, dann aber etwas mehr Zucker nehmen.\\
               Als Öl eignen sich besonder gut Walnussöl oder Rapsöl. Gut schmeckt auch Albaöl (Rapsöl mit Buttergeschmack).
    }
    
\end{recipe}
\begin{recipe}
[% 
    preparationtime = {\unit[45]{min}},
    bakingtime={\unit[30-35]{min}},
    bakingtemperature={\protect\bakingtemperature{
        fanoven=\unit[165]{\textcelcius}}},
        source = {Lieblingskuchen von Ilka}
]
{Rotweinkuchen mit Cranberries}
  
    \ingredients{%
        250g	& Butter \\	
		200g    & Zucker \\
		4		& Eier \\
		1 Schoppen & Rotwein\\
		250g	& Mehl\\
		50g 	& Cranberries\\
		50g		& gehackte Schokolade\\
		3 TL	& Kakao\\
		1 TL 	& Zimt\\
		1 P"ackchen & Backpulver\\
		230g 	& gemahlene N"usse\\
		250g 	& wei\ss e Kuvert"ure\\
    }
    
    \preparation{%
    \newline
       \step Butter mit Zucker und Eiern schaumig schlagen. Nach und nach Rotwein und 3 L"offel vom Mehl dazugeben. Den Rest Mehl mit Cranberries, gehackter Schokolade, Kakao, Zimt, Backpulver und gemahlenen N"ussen vermischen und unter die Buttermasse heben.
       \step Eine Gugelhupfform gut ausbuttern und mit Mehl best"auben, Kuchenmasse einf"ullen und ab damit in den Ofen.
       \step F"ur die Glasur die Kuvert"ure "uber dem Wasserbad schmelzen. Den erkalteten Kuchen damit bepinseln und mit Cranberries best"ucken.
    }
    \suggestion{}

    \hint{%
	      Schmeckt stets saftig und frisch.
    }
    
\end{recipe}
\begin{recipe}
[% 
     portion = {\portion{4}},
    source = {Lieblingsrezept von Sven (original Die Küchenschlacht)}
]
{Hausgemachte Tagliatelle mit Rinderfiletspitzen, Rucola und Tomaten}

    \ingredients{%
        600 g	& 	Rinderfiletspitzen \\
		200 g	& 	Rucola \\
		4 &	Schalotten \\
		2	&	Knoblauchzehen \\
		4 & Tomaten\\
2 & Zitronen\\
		 2 Bund & glatte Petersilie\\
		120 g & getrocknete Tomaten, eingelegt in Öl\\
		40 g& Pinienkerne\\
		100 g & Parmesankäse\\
		500 g & Mehl\\
		40 g & Hartweizengrieß\\
		4& große Eier\\
		& Zucker zum Abschmecken\\
		&Olivenöl zum Anbraten\\
					&	Butterschmalz zum Anbraten\\
			&	Pastagewürz zum Abschmecken \\
					&Schwarzer Pfeffer aus der Mühle\\
		& Salz aus der Mühle\\
    }
    
    \preparation{%
    \newline
       \step Das Mehl, den Hartweizengrieß und die Eier vermengen und zu einem Teig verarbeiten. Je nach Konsistenz
etwas Wasser zufügen. Anschließend den Teig ruhen lassen.
       \step Den Rucola waschen und die unteren Stiele abschneiden. Die Petersilienblätter waschen und vom Stängel
zupfen. Rucola und Petersilie zunächst in einer Schale zur Seite stellen.
       \step Die getrockneten Tomaten auf etwas Küchenpapier abtropfen lassen und in Streifen schneiden.
Die Pinienkerne in einer Pfanne ohne Öl anrösten. Eine Schalotte und einen Knoblauchzehe abziehen und
würfeln. Die Tomaten halbieren, vom Strunk befreien und fein hacken.
       \step Etwas Butterschmalz in einer weiteren Pfanne erhitzen und die Zwiebeln und den Knoblauch andünsten. Die
Pinienkerne, die Tomatenstreifen und Tomatenwürfel dazu geben. Die Zitrone waschen und halbieren.
Pinienkerne und Tomaten mit etwas
Zitronensaft und einer Prise Zucker abschmecken. Tomaten-
Pinienkernmischung auf niedriger Temperatur warm halten.
Anschließend den Parmesankäse reiben und zur Seite stellen.
     \step Für die Nudeln bereits einen Topf mit ausreichend Wasser aufsetzen und zum Kochen bringen. Den Nudelteig in
der Nudelmaschine zu Tagliatelle verarbeiten. Die Nudeln in dem kochenden Wasser mit etwas Salz gar kochen.
Anschließend abschütten.
 \step Die Rinderfiletspitzen waschen, trocken tupfen, mit Salz und Pfeffer würzen. Etwas Butterschmalz in einer Pfanne
erhitzen und das Rinderfilet kurz anbraten.
\step Den Rucola und die Petersilie kurz vor dem Anrichten in die Pfanne mit der Tomaten-Pinienkernmischung geben.
Nudeln und Filetspitzen dazugeben, alles schwenken und gut vermengen.
\step Die Tagliatelle mit den Filetspitzen, dem Rucola und den Tomaten auf einem Teller anrichten und mit dem
geriebenen Parmesan garnieren und servieren.
    }

\end{recipe}